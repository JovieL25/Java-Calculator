\documentclass[12pt]{article}
\begin{document}

\begin{titlepage}
	\begin{center}
	\huge{\bfseries Development of Transcendental Functions Calculator : Iteration I}\\
	[1.5cm]
	Team G\\
	[1.5cm]
	June, 3, 2020
	\end{center}
\end{titlepage}
\tableofcontents

\newpage
\section{Introduction}\label{sec:intro}
The goal for our project is to implement a calculator mainly focus on computing transcendental functions. This involves avoiding the use of math library which is available for most of the programming languages. Thus, the implementation of the project is rather complicated. Learning mathematics theorem and writing corresponding codes is essential for a succeed project.\\
[1cm]
\large{\bfseries What is Transcendental Functions:}\\
A transcendental function is an analytic function that does not satisfy a polynomial equation, in contrast to an algebraic function. In other words, a transcendental function "transcends" algebra in that it cannot be expressed in terms of a finite sequence of the algebraic operations of addition, multiplication, and root extraction.\\
\newpage
\section{Roles and Responsibilities}
\textit{As a result of team meeting, one person could be responsible for two roles.}\\
Team Manager: Xuan Li, the manager will organize team meeting, submit deliverables before the deadline, communicate with professor and TAs then give information to other team members.\\
Leading Programmers: Jingyi Lin, Yilu Liang, Derek Liu. The leading programmers are responsible for integrating codes also style of the final codes. Supervising and review pull requests in the team repository.\\
Documents Editors: Xuan li, Shiyu Lin, Zhen Long, Ziqian Li. The editors will be mainly writing reports that reflect the development progress of the calculator, in the meantime, recording the content of each team meeting, helping other teammates with their persona if needed.\\
QA/Tester: Derek Liu, Jingyi Lin, Yilu Liang. Apart from integrating codes, they also need to run tests and evaluate the codes.\\
Major Presenter: Derek Liu, the major presenter would represent the group to demonstrate to the professor work done during the given period of time.\\
Everyone in the team is responsible for at least one interview, and write out a persona.\\
Everyone has chosen a transcendental function that he/she will implement by using Java.\\
\newpage
\section{Interview Questions}
Type 1: information about the interviewee and knowledge about the survey

profession/education background/explanation of the topic, give examples if the interviewee is not clear with what a TF is/inform about the privacy issue(what it will be used for)\\

Type 2: Background/history related to usage, basic questions

1.Do you know about transcendental functions?or
Have you ever used transcendental functions in daily life?

2.In what cases do/did you use it?
 
3.What tools do you use to calculate?

4.Is (using that tool) common for people in your area, or you choose to use it?

5.frequency-how often do you calculate Transcendental Functions?\\

Type 3: The need of the user

1.What transcendental functions did you use?

2.What is the preferable precision for you?

3. What is the preferable interface for you? (textual, graphical)\\
 
Type 4: User experience

1.Do you find it easy to use? score 0-10

2.Is a manual necessary?Or do you prefer google when you find a problem/ when you were a newbie?

3.Can you recall how long does it take to get the result from typing in the numbers? Are you satisfied with that?

4.Is there any issues or the problems that you had while using a calculator?

5.What are your solutions to solve those problems?

6.Do you think the function that tool provided fulfills your need? (If not, what should be improved?)
 
7.Are you willing to turn to a specific designed tool for calculating if it fulfills all your needs?\\
 
Type 5:

1.Anyone else you know have the need of using it? Have you heard about their opinion? 
 
2.How important is a visual interface (GUI) to you when using a calculator?

3.What functions of a calculator do you use the most?

4.What functions of a calculator do you use the least?

5.Do you have a computer at work or at home while you need a calculator?

6.Are you familiar with command line application or programming?  (User must be able to use the java application and command line)

7.Are you willing to use a calculator application on a computer?

8.Do you agree that a scientific calculator should output the result as many digits as possible in terms of irrational numbers?

1.Disagree  2. Dont care 3. Agree 4. I don't know

9.Do you agree that a scientific calculator should take multiple inputs at once; in other words, is it better to take a function as input?

1.Disagree  2. Dont care 3. Agree 4. I don't know
\newpage
\section{Personas}

\newpage
\section{Use Cases}

\newpage
\section{Selected Functions}

\newpage
\section{Basic Mathematics Functions}

\newpage
\section{Glossary}

\newpage
\section{References}

\newpage
\section{Appendix Meeting Log}

\end{document}
